\documentclass[32pt,aspectratio=169]{beamer}

\usepackage[utf8]{inputenc} % Character encoding, choose your machine's default
                              % encoding; latin1 or utf8. If you want it read by
                              % others choose latin1.
\pdfinfo{
   /Author (Bashar Dudin)
   /Title  (Fixing Graph Theoretical Terminology)
   /Subject (Networks and Flows on Graphs)
}

\usepackage{./Style/My_Beamer} % This is my Beamer style.
\usepackage{./Style/Mystyle} % This is my own defined commands

%----------------------------------------------------------------------------------------
%   TITLE PAGE
%----------------------------------------------------------------------------------------

\author[BD]{Bashar Dudin}

\institute[]{EPITA}

\title{Networks and Flows on Graphs} %
\subtitle{Fixing Graph Theoretical Terminology}

\begin{document}

\begin{frame}[plain]
\titlepage % Print the title page as the first slide
\end{frame}

\begin{frame}{What Is a Graph?}
  \begin{defn}
    A \emph{directed graph} (or \emph{digraph}) $G$ is given by a set $V$ of
    \emph{vertices} together with a set $A$ of \emph{arrows}; an arrow being an
    (ordered) couple $a = (x,y)$ of vertices. The vertex $x$ is called
    the \emph{source} of $a$ while $y$ is its \emph{target}. We write $G = (V, A)$
    for the digraph $G$.
  \end{defn}
  \pause
  \begin{question}
    What is the most general digraph you can draw? Can you have two
    loops for a single edge? How many arrows in between two vertices?
  \end{question}
\end{frame}

\begin{frame}{What Is a Graph?}
  \begin{defn}
    An \emph{(undirected) simple graph} $G$ is given by a set $V$ of vertices
    together with a set $A$ of \emph{edges}; an edge being an (unordered)
    pair $e = \{x,y\}$ of vertices. We write $G = (V, E)$ for the
    digraph $G$.
  \end{defn}
  \pause
  \begin{rem}
    Notice that to be a \alert{pair} $\{x, y\}$ you need to have
    $x \neq y$. So, simple graphs do not have loops.
  \end{rem}

  \begin{overlayarea}{\textwidth}{.35\textheight}
    \begin{onlyenv}<3-4| handout:1>
      There is a canonical way of attaching a simple graph to a digraph
      $G$ :
      \begin{itemize}
      \item<3-> Delete loops of the set of arrows of $G$, these are given by
        couples having two identical entries
      \item<4-> build up the set of edges as the collectiong of pairs
        $\{x, y\}$ for each arrow $(x, y)$ in $G$.
      \end{itemize}
    \end{onlyenv}
    \begin{onlyenv}<5|handout:2>
      \begin{question}
        Can you think of others ways of defining a graph? For instance how
        would you give a definition to allow multiple arrows in a digraph?
        Or to allow loops and multiple edges in a graph?
      \end{question}
    \end{onlyenv}
  \end{overlayarea}
\end{frame}

\begin{frame}{Finding Your Way in Graphs}
  \begin{columns}
    \begin{column}{.55\textwidth}
      \begin{halfshyblock}{Graph}
        {\small
        \textbf{Chain :} A sequence of edges where each edge has a
        common vertex with the preceding one (except the first), the
        other being common with the next edge (except the last).

        \textbf{Cycle :} A closed chain.

        \textbf{Simple chain :} Containing each edge at most once.

        \textbf{Elementary chain :} Containing each vertex at most
        once.

        \textbf{Hamiltonian chain :} Passing once by each edge.

        \textbf{Eulerian chain :} Passing once by each vertex.

        \textbf{length of chain :} The number of edges in the chain.
      }
      \end{halfshyblock}
    \end{column}
    \begin{column}{.55\textwidth}
      \begin{halfshyblock}{Digraph}
        {\small \textbf{Path :} A sequence of arrows where each
          arrow's target is the source of the next arrow (except the
          last).

          \vspace{\baselineskip}

        \textbf{Circuit :} A closed path.

        \textbf{Simple path :} Containing each arrow at most once.

        \textbf{Elementary path :} Containing each vertex at most
        once.

        \textbf{Hamiltonian path :} Passing once by each arrow.

        \textbf{Eulerian path :} Passing once by each vertex.

        \textbf{length of chain :} The number of arrows in the chain.
      }
      \end{halfshyblock}
    \end{column}
  \end{columns}
\end{frame}

\begin{frame}{Connectedness}
\begin{columns}[T]
    \begin{column}{.55\textwidth}
      \begin{halfshyblock}{Graph}
        {\small
          \textbf{Connectedness :} A graph is said to be
          \emph{connected} if any two distinct vertices are the endpoints of
          a chain.

          \textbf{Connected components :} Maximal subgraphs that are
          connected.
          }
      \end{halfshyblock}
    \end{column}
    \begin{column}{.55\textwidth}
      \begin{halfshyblock}{Digraph}
        {\small
          \textbf{Strong connectedness :} A digraph is said to be
          \emph{strongly connected} if any two distinct vertices are the
          initial source and target of a path.

          \textbf{Strongly connected components :} Maximal subraphs
          that are strongly connected.}
      \end{halfshyblock}
    \end{column}
  \end{columns}
  \vspace{\baselineskip}
  \begin{rem}
    A digraph is said to be connected if its underlying graph is
    connected. Connected components of a digraph are defined the
    same way.
  \end{rem}
\end{frame}

\begin{frame}{Two extreme cases of connected graphs}
  \begin{columns}[T]
    \begin{column}{.55\textwidth}
      \begin{halfshyblock}{Trees}
        {\small
        A tree $T$ is a graph where each two vertices are linked by
        exactly one chain. It is equivalently given by
        \begin{itemize}
        \item $T$ is connected and cycle-free
        \item $T$ is connected of maximal order
        \item $T$ is connected and deleting any edge disconnects it
        \item $T$ is cycle-free and adding any edge creates one.
        \end{itemize}
        }
      \end{halfshyblock}
    \end{column}
    \begin{column}{.55\textwidth}
      \begin{halfshyblock}{Complete Graphs}
        {\small
        The complete graph $\mathcal{K}_n$ is the graph having $n$
        vertices and all possible edges linking them.

        \vspace{\baselineskip}

        It has exactly $\displaystyle{\frac{n(n-1)}{2}}$ edges.

        \vspace{\baselineskip}

        Can you draw $\mathcal{K}_5$ on a paper without having two
        edges overlapping?
        }
      \end{halfshyblock}
    \end{column}
  \end{columns}
\end{frame}

\begin{frame}{Adjacency Matrix}
  \begin{defn}
    Let $G = (V, E)$ be a graph. Two distinct vertices of $G$ are said
    to be adjacent if they are endpoints of the same edge.
  \end{defn}
  \pause One can represent a graph in a machine in one of the two
  following ways:
  \begin{itemize}
  \item<3->[\textcolor<5->{lightgray!30!white}{\textbullet}]
  \textcolor<5->{lightgray!30!white}{As a collection of vertices, each
    coming with its list of adjacent vertices}
  \item<4-> As a matrix called the \emph{adjacency matrix} of the graph.
  \end{itemize}
\end{frame}

\begin{frame}{Adjacency Matrix}
  \begin{defn}
    Let $G$ be a graph (resp. digraph) having $n$ vertices. The
    \emph{adjacency matrix} of $G$ is a square matrix having $n$
    columns and $n$ rows, identically indexed by the vertices. Entry
    $(i, j)$ is $1$ iff there is an edge (arrow) from $i$ to $j$,
    otherwise it is zero.
  \end{defn}
  \pause
  \begin{rem}
    The adjacency matrix of a graph is symmetric. A digraph having a
    symmetric adjacency matrix has, for any given arrow $(x, y)$
    between distinct vertices, an arrow $(y, x)$ going the other way
    around. \pause That's the reason why some mathematical textbooks see a
    graph as a digraph having this property. In that context arrows
    are called half-edges.
  \end{rem}
  \pause
  \begin{question}
    Do you know any interesting theoretical results about symmetric
    matrices?
  \end{question}
\end{frame}

\begin{frame}{Adjacency Matrix}
  \begin{prop}
    Let $G$ be either a graph or a digraph and write $M$ for its
    adjacency matrix. For any given $k \in \N^*$, the matrix $M^k$ has
    entry $(i, j)$ equal to the number of chains (paths) having source
    $i$ and target $j$.
  \end{prop}
  \pause
  \setlength\columnseprule{.1pt}
  \begin{multicols}{2}
    \begin{demo}
      This is done by induction. In case $k = 1$, a path of length $1$
      is just an edge (or an arrow) and this is just the definition of
      the adjacency matrix. \pause Assume that entries of the matrix
      $M^k$ correspond to the number of chains (paths) from the row to
      the column index. Let $a_k[i, j]$ be the $(i, j)$ coefficient of
      $M^k$. Then the entry $a_{k+1}[i, j]$ of $M^{k+1}$ is given by
      \begin{align*}
        a_{k+1}[i, j] &= \sum_{\ell = 1}^n a_k[i, \ell]a_1[\ell, j]\\
                      & = \sum_{\big\{\ell \big| \textrm{$\ell$ is adjacent to $j$}\big\}} a_k[i, \ell]
      \end{align*}
      which is exactly what we are looking for.
    \end{demo}
  \end{multicols}
\end{frame}

\begin{frame}{Generalization and Variant of Adjacency Matrix}
  \begin{defn}
    Let $G$ be a graph (resp. digraph) having $n$ vertices and
    \emph{weighted} edges (resp. arrows). The \emph{adjacency matrix}
    of $G$ is a square matrix having $n$ columns and $n$ rows,
    identically indexed by the vertices. If there is an edge from $i$
    to $j$ then entry $(i, j)$ gets the weight of that edge, otherwise
    entry $(i, j)$ is zero.
  \end{defn}
  \pause Given such a graph $G$ with possible weights different from
  $1$, then the $k$-th power of its adjacency matrix $M$ is less
  easily interpretable; the contribution of each edge or arrow to a
  path will be taken with the corresponding weight. \pause

  \vspace{\baselineskip}

  Sometimes we are only interested in the fact there is an edge,
  arrow, chain or path between two given vertices. In that case we
  look at the adjacency matrix as a \emph{boolean} matrix. This will
  be worked out in an exercice later on.
\end{frame}

\begin{frame}
  \begin{center}
    {\Large \textbf{It's all for now!}}
   \end{center}
 \end{frame}


%%%%%%%%%%%%%%%%%%%%%%%%%%%%%%%%%%%%%%%%%%%%%%%%%%%%%%%%%%%%%%%

\end{document}

%%% Local Variables:
%%% mode: latex
%%% TeX-master: t
%%% End:

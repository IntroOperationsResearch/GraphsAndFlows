\documentclass[32pt, aspectratio=169]{beamer}

\usepackage[utf8]{inputenc} % Character encoding, choose your machine's default
                              % encoding; latin1 or utf8. If you want it read by
                              % others choose latin1.
\pdfinfo{
   /Author (Bashar Dudin)
   /Title  (Maximum Flow Problem)
   /Subject (Networks and Flows on Graphs)
}

\usepackage{./Style/My_Beamer} % This is my Beamer style.
\usepackage{./Style/Mystyle} % This is my own defined commands

%----------------------------------------------------------------------------------------
%   TITLE PAGE
%----------------------------------------------------------------------------------------

\author[BD]{Bashar Dudin}

\institute[]{EPITA}

\title{Networks and Flows on Graphs} %
\subtitle{Maximum Flow Problem}

\begin{document}

\begin{frame}[plain]
\titlepage % Print the title page as the first slide
\end{frame}

\begin{frame}{What is it about?}
  \begin{halfshyblock}{Maximum Flow Problem}
    You have a water network together with a unique source. Each water
    conduct has a maximum water flow capacity. Using the whole network
    how can you maximize the water flow at a given sink?
  \end{halfshyblock}
\end{frame}

\begin{frame}{Mathematical Modeling}
  A (transportation) network is a simple digraph $G = (V, A)$ having a
  single source $s$, a single sink $t$, and for each arrow $a$ of $G$
  a valued capacity $c(a)$.
  \pause
  \begin{defn}
    A flow on a network $G$ is a map $f: A \to \R_+$ such that:
    \begin{itemize}
    \item For each $a \in A$, $f(a) \leq c(a)$,
    \item For each vertex $v$, except for the source and sink, the sum
      of values $f$ takes on ingoing arrows is equal to the sum taken
      by $f$ on outgoing arrows. In a more pedantic fashion
      \begin{displaymath}
        \forall v \in V\setminus \{s, t\}, \quad \sum_{\{a | a \to s\}} f(a) = \sum_{\{a | s \to a\}}f(a)
      \end{displaymath}
    \end{itemize}
  \end{defn}
\end{frame}

\begin{frame}{Mathematical Modeling}
  \begin{defn}
    The value $\val\big[f\big]$ of a flow $f$ on a network $G = (V, A)$ is the
    sum of values taken by $f$ on arrows going into $t$.
  \end{defn}
  \pause
  \begin{halfshyblock}{Maximum Flow Problem}
    Given a network $G$ find a flow $f$ on $G$ with maximum value
    flow.
  \end{halfshyblock}
\end{frame}

\begin{frame}{Cuts}
  \begin{defn}[of a cut]
    A cut of a digraph $G=(X, A)$ is a subset $S$ of vertices such that $S$
    contains the source $s$ but not the sink $t$.
  \end{defn}
  \pause
  We write $\overline{S}$ for the complement of $S$ in $X$, the set
  of arrows having tail in $S$ and head in $\overline{S}$ is written
  $\big(S, \overline{S} \big)$.
  \begin{rem}
    The two simplest cuts are respectively given by the cut only
    containing the source and the one containing every vertex but the sink.
  \end{rem}
  \pause
  \begin{defn}
    Given a flow $f$ on a network $G= (X, A)$. The value flow
    $f\big[S\big]$ at a cut $S$ of $G$ is the sum
    $f\big[S, \overline{S} \big]$ of values taken by $f$ on elements in
    $\big(S, \overline{S}\big)$ minus the sum $f\big[\overline{S}, S\big]$
    of values taken by $f$ on $\big(\overline{S}, S\big)$.
  \end{defn}
\end{frame}

\begin{frame}{Cuts}
  \begin{prop}
    Given a flow $f$ on a network $G$ the value flow at any cut
    is independant of the cut.
  \end{prop}
  \begin{overlayarea}{\textwidth}{.65\textheight}
    \begin{onlyenv}<2| handout:1>
      \vspace{2pt}
      This means the value of a given flow can be recovered as the
      value at any cut. In particular, the value flow is equal to the
      sum of values taken by the flow at arrows going out of the
      source.
    \end{onlyenv}

    \begin{onlyenv}<3-| handout:2>
      \vspace{2pt}
      \begin{demo}
    Let $S_\infty$ be the cut of $G$ containing everything but the
    sink. Let
    $S \subsetneq S_1 \subsetneq \cdots \subsetneq S_\infty$ be a
    maximal sequence of cuts (convince yourself that $S_{k+1}$ has
    exactly one more vertex than $S_k$). By definition
    \begin{displaymath}
      f\big[S_\infty\big] = \val\big[f\big]
    \end{displaymath}
    Let $v_{k+1}$ be the only vertex in $S_{k+1}$ but not in $S_k$.
    \begin{displaymath}
      f\big[S_{k+1}\big]  =
      f\big[S_{k}\big] - \sum_{S_{k} \to a \to v_{k+1}} f(a)
      + \sum_{v_{k+1} \to a \to S_{k}} f(a)
       + \sum_{v_{k+1} \to a \to \smash{\overline{S}}_{k+1}} f(a)
      - \sum_{\smash{\overline{S}_{k+1}}\to a \to v_{k+1}} f(a)
    \end{displaymath}
    These last four terms are exactly the difference between the
    values of $f$ at ingoing arrows into $v_{k+1}$ and outgoing ones
    from $v_{k+1}$. This difference is zero.
  \end{demo}
\end{onlyenv}
\end{overlayarea}
\end{frame}

\begin{frame}{Cuts}
  \begin{defn}
    Given an s-t cut $S$ of $G$, we call \emph{capacity} of $S$ the
    non-negative number
    \begin{displaymath}
      c\big[S\big] = \sum_{a \in (S, \smash{\overline{S}})} c(a).
    \end{displaymath}
  \end{defn}
  \begin{halfshyblock}{Facts}
    Let $f$ be a flow on a network $G$ and let $S$ be an s-t cut of
    $G$.
    \begin{itemize}
      \item $\val\big[f\big] \leq c\big[S\big]$
      \item $\val\big[f\big] = c\big[S\big]$ iff for all
        $a \in \big(S, \overline{S}\big)$, $f(a) = c(a)$ and for all
        $a \in \big(\overline{S}, S\big)$, $f(a) = 0$.
    \end{itemize}
  \end{halfshyblock}
  Thus, if $\val\big[f\big] = c\big[S\big]$ then $f$ is a flow having
  maximal flow value and $S$ is an s-t cut having minimal capacity.
\end{frame}
\begin{frame}{Max-Flow Min-Cut Theorem}
\begin{thm}
    The maximum value over all s-t flows is equal to the minimum
    capacity over all s-t cuts.
  \end{thm}
  \pause
  A more precise statement is to say that, given a network $G$, there
  is a flow $f$ and an s-t cut $S$ such that
  $\val\big[f\big] = c\big[S\big]$ iff $f$ has maximum value among
  flows on $G$ and $S$ has minimum capacity among s-t cuts of $G$.
  \pause
  \begin{halfshyblock}{Conclusion}
    Finding the maximum value of a flow is equivalent to finding a
    minimizing cut.
  \end{halfshyblock}
\end{frame}

\begin{frame}{Ford-Fulkerson Algorithm}
  A chain $C$ in a network $G$, joining the source $s$ to the sink
  $t$, comes with a natural orientation (a choice of an arrow for each
  edge) consistantly going from the source to the sink. We write $C_+$
  the set of arrows of $G$ supported on $C$ and matching its
  orientation. We write $C_-$ the set of arrows of $G$ supported on
  $C$ and opposite to that orientation.  \pause
  \begin{defn}
    Let $f$ be a flow on a network $G$. An \emph{augmenting chain} in $G$ is a chain
    $C$ joining the source $s$ to the sink $t$, such that
    \begin{displaymath}
      \alpha = \min\Big\{\min_{a \in C_+}c(a)-f(a), \min_{a \in C_-}f(a)\Big\} > 0
    \end{displaymath}
  \end{defn}
\end{frame}

\begin{frame}{Ford-Fulkerson Algorithm}
  Let $f$ be a flow on a network $G$. Given an augmenting chain $C$ in
  $G$ one can modify $f$ along $C$ (and $C$ alone) by adding $\alpha$
  to each arrow in $C_+$ and substracting $\alpha$ to each arrow in
  $C_-$. Looking at the value of this new flow $\varphi$ at the source
  one can see that
  \begin{displaymath}
    \val[\varphi] = \val[f] + \alpha
  \end{displaymath}
  This shows that $f$ was not a flow of maximim flow value. \pause In fact
  \begin{prop}
    A flow is of maximum flow value on a network $G$ iff there are no
    augmenting chains in $G$.
  \end{prop}
  \pause
  \begin{rem}
    The fact a flow on a network that doesn't have any augmenting
    chains is of maximal flow value is harder to grasp. You can get a
    proof in \emph{Introduction to Algorithms} by Thomas H.~Cormen and
    al.
  \end{rem}
\end{frame}

\begin{frame}{Ford-Fulkerson Algorithm}
  \begin{algorithm}[H]
    \caption{Ford-Fulkerson Algorithm}
    \small{
      \begin{algorithmic}[1]
       \Statex
       \Require $G$ a network and $f$ a flow on $G$
       \Ensure a maximal flow on $G$
       \Statex
       \Procedure{Ford-Fulkerson}{$G$, $f$}
       \State $(C, \alpha) \gets$ \Call{AugmentingChain}{$G$, $f$}
       \While{$C \neq$ \textbf{None}}
       \State $f$ is increased or decreased by $\alpha$ on arrows in $C$ depending on orientation
       \State $(C, \alpha) \gets$ \Call{AugmentingChain}{$G$, $f$}
       \EndWhile
       \EndProcedure
       \Statex
     \end{algorithmic}
     }
    \end{algorithm}
\end{frame}

\begin{frame}{Ford-Fulkerson Algorithm}
  \begin{algorithm}[H]
    \caption{Finding an augmenting chain}
    \small{
    \begin{algorithmic}[1]
      \Statex
      \Require $G$ a network and $f$ a flow on $G$
      \Ensure an augmenting chain on $G$ if any, and $\alpha$
      \Statex
      \Function{AugmentingChain}{$G$, $f$}
      \State Mark the source by a $+$
      \State Mark head $h$ of any unsaturated arrow $(t, h)$, whose tail is marked, by $+t$
      \State Mark tail $t$ of any arrow $(t, h)$ having non-zero flow, whose head is marked, by $-h$
      \If{sink is marked}
      \State \Return an augmenting chain $C$ by following marks from the sink to the source, and $\alpha$
      \EndIf
      \EndFunction
      \Statex
    \end{algorithmic}
    }
   \end{algorithm}
\end{frame}

\begin{frame}{Termination}
  \begin{halfshyblock}{Termination}
    The Ford-Fulkerson algorithm terminates if all capacities and flow
    values on arrows are rational.
  \end{halfshyblock}\pause
  This is clear : if flow is not of maximal value, we can find an
  augmenting chain along which we can increase flow value. At some
  point we get a maximal flow. \pause As to why we get a flow of
  maximal value : launch Ford-Fulkerson's algorithm one last time,
  write $S$ for all marked vertices (this is not the set of all
  vertices because there is no augmenting chains), then the set of all
  arrows in $\big(S, \overline{S}\big)$ are saturated and all those in
  $\big(\overline{S}, S\big)$ get flow value $0$. This is enough to
  say that the capacity of s-t cut $S$ is equal to the value flow we
  got; i.e. we have a minimal cut and thus a maximal flow.
  \begin{rem}
    If you google the Ford-Fulkerson algorithm, you should be able to
    find counter-examples in the case of irrational capacities.
  \end{rem}
\end{frame}

\begin{frame}{Complexity}
  The complexity of the Ford-Fulkerson algorithm depends on how
  efficient you are to find augmentig paths, that's why it is
  sometimes called the Ford-Fulkerson \emph{method} rather than
  algorithm. \pause Without specifying how we visit vertices we have
  that
  \begin{halfshyblock}{Complexity}
    Given a network $G = (V, A)$ with integer capacities and maximal
    value flow $m$ the Ford-Fulkerson complexity is $O\big(ma\big)$,
    where $a$ is the number of arrows of $G$.
  \end{halfshyblock}
  To increase flow value by $1$ you need to find an augmenting path by
  going through arrows of $G$, to possibly mark their sources and
  targets. Starting by the $0$ flow you need to do it at worst $m$
  times, each time going through all arrows. \pause
  \begin{question}
    Can you imagine modifying the given algorithm for a better
    complexity?
  \end{question}
\end{frame}

\begin{frame}{Further readings}
  We are done with the basics, we'll be going through a couple of
  exercices about maximal flows and matchings. \pause Your are
  encouraged to read about the following two related topics
  \begin{itemize}
  \item maximal flows at minimal costs
  \item general preference based matchings (uses what is sometimes
    called the \emph{hungarian method}).
  \end{itemize}
\end{frame}

%%%%%%%%%%%%%%%%%%%%%%%%%%%%%%%%%%%%%%%%%%%%%%%%%%%%%%%%%%%%%%%

\end{document}


  % Notice that given a flow $f$ on a network $G$ the value of $f$ is
  % equally the sum of the value flow on arrows going out of the source
  % (minus the value at the source if there is any). \pause Here is a
  % way to convince yourself of this fact: \pause Given a vertex $v$ of $G$
  % call a predecessor of $v$ the tail of any arrow going into $v$. The
  % value flow of $f$ is the sum of values taken at arrows going into
  % the sink $t$. The balancing condition at each vertex says this is
  % also the sum of values taken by $f$ at arrows going into each
  % predecessor of $t$. If you're careful enough about termination of
  % such process you can prove the previous claim. It will be proven in
  % a more general context.

%%% Local Variables:
%%% mode: latex
%%% TeX-master: t
%%% End:
